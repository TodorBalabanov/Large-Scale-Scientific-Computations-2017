\documentclass{llncs}
%
\begin{document}
\pagestyle{headings} 
\mainmatter
%
\title{2D Optimal Packing with Population Based Algorithms}
%
\author{Desislava Koleva\inst{1}, Maria Barova\inst{2}, Petar Tomov\inst{2}}
%
\authorrunning{Desislava Koleva} 
%
\tocauthor{Desislava Koleva, Maria Barova, Petar Tomov}
%
\institute{University College London, Department of Computer Science, Gower Street, London, WC1E 6BT, United Kingdom\\
\email{desislava.koleva.15@ucl.ac.uk}
\and
Institute of Information and Communication Technologies, Bulgarian Academy of Sciences, acad. G. Bonchev Str, Block 2, 1113 Sofia, Bulgaria}
%
%Desislava Koleva desislava.koleva.15@ucl.ac.uk
%Maria Barova maria_b88@abv.bg
%Petar Tomov petyr.tomov@gmail.com 
%
\maketitle
%
\begin{abstract}
This study addresses application of population based optimization heuristics to the solution of packing problems as part of optimal cutting tasks in the field of operations research. Such problems are very common in the industrial material cutting. It has one, two or three dimensional variations. The focus of this paper is on two dimensional case of steel sheet cutting. A description of two dimensional plates is supplied as algorithm input. The output of the algorithm is coordinates of the plates in the steel sheet and angle of rotation for each plate. Population based global optimization heuristics are used for optimal packing. All experiments are done with open source libraries for 2D geometry and population based heuristics. 
\keywords{Optimal Cutting Problem, Optimal Packing, Evolutionary Algorithms, Optimization}
\end{abstract}
%
\section{Introduction}
%
Optimal packing problem is an optimization problem in mathematics that involves attempting to pack objects together into container (or many containers). One of the usual goals is to pack a single container as densely as possible. This problem is related to real life packaging (in fact optimal cutting) problem as described in [1,2]. This study is related with the problem presented at ESGI120 [2] and ESGI113 [4]. The goal is to cut optimally on pieces a sheet of steel and because of that shapes overlapping is not allowed. The cutting result shapes are irregular not self-intersecting polygons. The problem presented at ESGI113 was less complicated than the problem presented at ESGI120, because it was rectangles packing in rectangle sheet of sepcified material. Similar problem is well presented in [5]. With irregular shapes and when the orientation of the given shape is not a constraint, the general nesting approaches are not particularly successful [9]. 

This section stars with problem description and refers to related works in a brief review. Section 2 points out the geometric considerations necessary to understand
this work. Section 3 describes the underlying rules and criteria used to build the GA based heuristic approach, which is proposed in the same section. In Section 4 some evaluation of the quality of the proposed approach is defined and some results obtained are presented. Finally, the last section draws conclusions and some comments about future work are made.
%
\section{Geometric considerations}
%
All shapes are represented as polygons. Arcs are approximated with small lines. Holes inside the shapes are not considered by definition [2]. Each polygon
is represented as set of vertices [3]. It is not allowed polygons to overlap. All polygons must be entirely placed inside the stock sheet.
%
\section{Genetic Algorithm for Optimal Packing Order}
%
%
\section{Experiments and Results}
%
%
\section{Conclusions}
%
%
\section*{Acknowledgements}
%
%
This work was supported by private funding of Velbazhd Software LLC.
%
% ---- Bibliography ----
%
\begin{thebibliography}{}
%
\bibitem {evti:fid}
Evtimov, G., Fidanova, S.:
Ant Colony optimization algorithm for 1D Cutting Stock Problem.
Proceedings of 11th Annual Meeting of the Bulgarian Section of SIAM, FASTUMPRINT, Sofia, Bulgaria, 24--25 (2016)
%
\bibitem {evti}
Evtimov, G.:
Project 2: Optimal cutting problem.
STOBET Ltd., 120th European Study Group with Industry, Sofia, Bulgaria  (2016)
%
\bibitem {bal:evti}
Balabanov, T., Evtimov, G., Koleva, D.:
ESGI 120 - Problem 2 - Genetic Algorithm Solver.
https://github.com/VelbazhdSoftwareLLC/ESGI120Problem2GeneticAlgorithmSolver Sofia, Bulgaria  (2016)
%
\bibitem {avdz:bal}
Avdzhieva, A., Balabanov, T., Evtimov, G., Kirova, D., Kostadinov, H., Tsachev, Ts., Zhelezova, S., Zlateva N.:
Optimal Cutting Problem.
Problems \& final reporst of 113-th European Study Group with Industry, FASTUMPRINT, Sofia, Bulgaria, 49--61 (2015)
%
\bibitem {mart:mon}
Martelloa, S., Monacib, M.:
Models and algorithms for packing rectangles into the smallest square.
Computers \& Operations Research, vol. 63, 161--171 (2015)
%
\bibitem {bal:1}
Balabanov, T.:
Distributed evolutional model for music composition by human-computer interaction.
Proceedings of International Scientific Conference UniTech15, University publishing house V. Aprilov, Gabrovo, Bulgaria, vol. 2, 389--392 (2015)
%
\bibitem {bal:2}
Balabanov, T.:
Avoiding Local Optimums in Distributed Population based Heuristic Algorithms (in Bulgarian).
Proceedings of XXIII International Symposium Management of energy, industrial and environmental systems, John Atanasoff Union of Automation and Informatics, Sofia, Bulgaria, 83--86 (2015)
%
\bibitem {bal:3}
Balabanov, T.:
Heuristic Forecasting Approaches in Distributed Environment (in Bulgarian).
Proceedings of Anniversary Scientific Conference 40 Years Department of Industrial Automation, UCTM, Sofia, Bulgaria, 163--166 (2011)
%
\bibitem {}
Teresa Costa, M.,  Miguel Gomes, A., Oliveira, J.:
Heuristic approaches to large-scale periodic packing of irregular shapes on a rectangular sheet.
European Journal of Operational Research, vol. 192, 29--40 (2009)
%
\end{thebibliography}
\end{document}
